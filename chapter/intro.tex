From long ago till recently, people are devoted to the discuttion and investigation of improving the performances of polymer product to meet their different desire. By adding nano- or micro-sized fillers to the polymer matrix, we found thus we are able to change the polymers' mechanical, optical, chemical or electrical properties.~\cite{Kutvonen2012,Humphrey1996a,Plimpton1995,Yoon2002,Chapman2001,Schmidt2003} But the mechanism of the reinforcement of the Polymer Nano-sized-filler Composite (PNC) are still elusive. It is widely acknowledged that there are many influencing factors to this matter, such as, the shape and the size of the filler, the molecular weight of the polymer, the degree of crosslinking of the polymer matrix, the strength of the interaction between the polymer and the filler, the mass-loading of the filler, and dispersity of the filler in the polymer matrix. As the development of the Nano-technology, we already get some advance from both experimental and theoretical points of view, however, the response of NPC to the compression and the shearing, and the mechanism behind are not very well studied and need more profound insigts. 

We now are able to use computer simulation to study PNC systems at the atom scale in detail, taking advantage of fast growth of the computer science and technology. Dilip Gersappe depicted a Coarse-Grained (CG) Model of PNC system in 2002, and investiaged the PNC system responses to the tensile~\cite{Gersappe2002}. Aki Kutvonen and etc. then studied the tensile of the PNC system using a CG model, and demonstrated that the size, the shape, the mass loading and the surface area of the filler will affect the properties of the PNC system.~\cite{Kutvonen2012,Kutvonen2012a} Many other models are then been designed in different scales trying to simulate the tensile, the compression of the PNC systems, and to analyze the inflences of the dispersity of the filler, the strength of the interaction between the polymer and the filler, and other factores, to the properties and the performances of the PNC system.~\cite{Gersappe2002,Liu2011a,Allegra2008,Brown2008,Brown2003} 

Experimental and theoretical models are developed fast and have been studied widely, however, the compression and the shearing of the PNC system are seldomly investigated and the mechanism is still under consideration. We thus simulate the compression procedure of the PNC system, using a CG model built and manipulatd by the molecular dymamics soft package LAMMPS~\cite{Wu2011}, to investigate the response of PNC to the compression, and describe the mechainism of the reinforcement by the filler to the system. All our analysis of the result are with the help of the software Visual Molecular Dynamics (VMD)~\cite{Hussain2006} 
