% Under Creative Commons Attribution licence 3.0
% (http://creativecommons.org/licences/by/3.0)
% Author: Florian Lesaint
\documentclass[landscape]{article}
\usepackage{tikz}
%%%<
\usepackage{verbatim}
\usepackage[active,tightpage]{preview}
\PreviewEnvironment{tikzpicture}
\setlength\PreviewBorder{10pt}%
%%%>
\begin{comment}
:Title: Cuboid in a 2 vanishing points perspective
:Tags: 3D;Geometry;Mathematics
:Author: Florian Lesaint
:Slug: cuboid

This code draws a cuboid using a 2 vanishing points perspective.
Within the code, parameters can be revised to tune the drawing.
\end{comment}
\usetikzlibrary{calc}

%\newcount\mycount
\begin{document}
\begin{tikzpicture}
	%%% Edit the following coordinate to change the shape of your
	%%% cuboid
      
	%% Vanishing points for perspective handling
	\coordinate (P1) at (-7cm,1.5cm); % left vanishing point (To pick)
	\coordinate (P2) at (8cm,1.5cm); % right vanishing point (To pick)
	\coordinate (P3) at (-7cm,-3cm); % left vanishing point (To pick)
	\coordinate (P4) at (-1cm,8cm); % right vanishing point (To pick)

	%% (B1) and (B2) defines the 2 central points of the cuboid
	\coordinate (B1) at (0em,0cm); % central top point (To pick)
	\coordinate (B2) at (0em,-2cm); % central bottom point (To pick)

	%% (A2) to (A8) are computed given a unique parameter (or 2) .8
	% You can vary .8 from 0 to 1 to change perspective on left side
	\coordinate (A2) at ($(P1)!.8!(B2)$); % To pick for perspective point 2
	\coordinate (A1) at ($(P1)!.7!(B1)$); % point 1

	% You can vary .8 from 0 to 1 to change perspective on right side
	\coordinate (A4) at ($(P2)!.7!(B2)$); % point 4
	\coordinate (A8) at ($(P2)!.7!(B1)$); 

	%% Automatically compute the last 2 points with intersections
	\coordinate (A5) at
	  (intersection cs: first line={(A8) -- (P1)},
			    second line={(A1) -- (P2)});
	\coordinate (A3) at
	  (intersection cs: first line={(A4) -- (P1)}, 
			    second line={(A2) -- (P2)}); % point 3

	%%% Depending of what you want to display, you can comment/edit
	%%% the following lines

	%% Possibly draw back faces

	\fill[gray!90] (B2) -- (A2) -- (A3) -- (A4) -- cycle; % face 6
	%\node at (barycentric cs:B2=1,A2=1,A3=1,A4=1) {\tiny n2};
	
	\fill[gray!50] (A2) -- (A1) -- (A5) -- (A3) -- cycle; % face 3
	%\node at (barycentric cs:A2=1,A1=1,A5=1,A3=1) {\tiny n1};
	
	%\fill[gray!30] (A5) -- (A3) -- (A4) -- (A8) -- cycle; % face 4
	%\node at (barycentric cs:A5=1,A3=1,A4=1,A8=1) {\tiny f4};
	
	%\draw[thick,dashed] (A5) -- (A3);
	\draw[->][thick,solid] (A2) -- (A3);
	\draw[->][thick,solid] (A3) -- (A4);

	%% Possibly draw front faces

	% \fill[orange] (B1) -- (A8) -- (A4) -- (B2) -- cycle; % face 1
	% \node at (barycentric cs:B1=1,A8=1,A4=1,B2=1) {\tiny f1};
	%\fill[gray!50,opacity=0.2] (B1) -- (B2) -- (A2) -- (A1) -- cycle; % f2
	%\node at (barycentric cs:B1=1,B2=1,A2=1,A1=1) {\tiny f2};
	%\fill[gray!90,opacity=0.2] (B1) -- (A1) -- (A5) -- (A8) -- cycle; % f5
	%\node at (barycentric cs:B1=1,A1=1,A5=1,A8=1) {\tiny f5};

	%% Possibly draw front lines
	%\draw[thick] (B1) -- (B2);
	\draw[->][thick,solid] (A1) -- (A2);
	%\draw[thick] (A4) -- (A8);
	%\draw[thick] (B1) -- (A1);
	%\draw[thick] (B1) -- (A8);
	%\draw[thick] (B2) -- (A2);
	%\draw[thick] (B2) -- (A4);
	%\draw[thick] (A1) -- (A5);
	%\draw[thick] (A8) -- (A5);
	
	\coordinate (R1) at ($(A1)!.5!(A2)$); % To pick for perspective 
	\coordinate (R2) at ($(A2)!.5!(A3)$);
	\coordinate (R3) at ($(A3)!.5!(A4)$);
	\coordinate (NS1) at (barycentric cs:A2=1,A1=1,A5=1,A3=1);
	\coordinate (NS2) at (barycentric cs:B2=1,A2=1,A3=1,A4=1);
	\coordinate (NE1) at ($(P3)!1.1!(NS1)$); % To pick for perspective 
	\coordinate (t1) at ($(P3)!1.16!(NS1)$); % To pick for perspective 
	\coordinate (NS12) at ($(P3)!1.105!(NS1)$); % To pick for perspective 
	\coordinate (NE12) at ($(P3)!1.2!(NS1)$); % To pick for perspective 
	\coordinate (NE2) at ($(P4)!1.05!(NS2)$);
	\coordinate (t2) at ($(P4)!0.87!(NS2)$); % To pick for perspective 
	\coordinate (NE22) at ($(P4)!0.78!(NS2)$); % To pick for perspective 
	\draw[->][solid] (NS1) -- (NE1);
	\draw[dashed] (NS12) -- (NE12);
	\draw[->][dotted] (NS2) -- (NE2);
	\draw[dashed] (NE22) -- (NS2);

	\coordinate (CP) at
	  (intersection cs: first line={(NE1) -- (NE12)}, 
			    second line={(NS2) -- (NE22)});

    %\draw[domain=0:90] (CP) circle (0.16); 
    %\draw (t1) .. controls ($(CP)+(0.2,0.2)$) .. (t2);  
    \draw[<->] ($(CP)+(-83:2mm)$) arc (-83:23:2mm);

	\draw[fill=black] (A1) circle (0.05em)
	    node[above right] {\tiny 1};
	\draw[fill=black] (A3) circle (0.05em)
	    node[above right] {\tiny 3};
	\draw[fill=black] (A4) circle (0.05em)
	    node[above right] {\tiny 4};
	\draw[fill=black] (A2) circle (0.05em)
	    node[above right] {\tiny 2};
	\draw[fill=black] (R1) circle (0.025em)
	    node[right] {\tiny r1};
	\draw[fill=black] (R2) circle (0.025em)
	    node[above] {\tiny r2};
	\draw[fill=black] (R3) circle (0.025em)
	    node[above right] {\tiny r3};
	\draw[fill=black] (NS1) circle (0.05em)
	    node[right] {\tiny n1};
	\draw[fill=black] (NS2) circle (0.05em)
	    node[below right] {\tiny n2};
	\draw[fill=black] (CP) circle (0.05em)
	    node[below right] {\tiny $\theta$};
	%\draw[fill=black] (t1) circle (0.05em)
	%    node[right] {\tiny x1};
	%\draw[fill=black] (t2) circle (0.05em)
	%    node[right] {\tiny x2};
	% \draw[fill=black] (P1) circle (0.1em) node[below] {\tiny p1};
	% \draw[fill=black] (P2) circle (0.1em) node[below] {\tiny p2};
\end{tikzpicture}
\end{document} 
