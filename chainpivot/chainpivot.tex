% Under Creative Commons Attribution licence 3.0
% (http://creativecommons.org/licences/by/3.0)
% Author: Florian Lesaint
\documentclass[landscape]{article}
\usepackage{tikz}
%%%<
\usepackage{verbatim}
\usepackage[active,tightpage]{preview}
\PreviewEnvironment{tikzpicture}
\setlength\PreviewBorder{10pt}%
%%%>
\begin{comment}
:Title: Cuboid in a 2 vanishing points perspective
:Tags: 3D;Geometry;Mathematics
:Author: Florian Lesaint
:Slug: cuboid

This code draws a cuboid using a 2 vanishing points perspective.
Within the code, parameters can be revised to tune the drawing.
\end{comment}
\usetikzlibrary{calc}

%\newcount\mycount
\begin{document}
\begin{tikzpicture}
	%%% Edit the following coordinate to change the shape of your
	%%% cuboid
      
	%% Vanishing points for perspective handling
    \coordinate (O) at (0cm,0cm); % left vanishing point (To pick)
    \coordinate (Xn) at (-1cm,0cm); % right vanishing point (To pick)
	\coordinate (Xp) at (1.5cm,0cm); % right vanishing point (To pick)
	\coordinate (YV) at (-1.732cm,-1cm); % left vanishing point (To pick)
	\coordinate (Yn) at ($(YV)!0.6!(O)$); % right vanishing point (To pick)
	\coordinate (Yp) at ($(YV)!1.5!(O)$); % right vanishing point (To pick)
	\coordinate (Zn) at (0cm,-1cm); % right vanishing point (To pick)
	\coordinate (Zp) at (0cm,1.5cm); % right vanishing point (To pick)
    
    \draw[fill=black] (O) circle (0.01em) node[above left] {\tiny O};
    \draw[->][thin,gray!80] (Xn) -- (Xp) node[right,gray] {\tiny x};
    \draw[->][thin,gray!80] (Yn) -- (Yp) node[above] {\tiny y};
    \draw[->][thin,gray!80] (Zn) -- (Zp) node[left] {\tiny z};

    \coordinate (A) at (-1cm,-1cm); % left vanishing point (To pick)
    \coordinate (B) at (1cm,1cm);
    \coordinate (Bxy) at (1cm,0.3cm);
    \coordinate (C) at (-0.5cm,0.5cm);
    \coordinate (D) at (0cm,0cm);
    \draw[solid,thick] (B) -- (A);
    \draw[fill=black] (A) circle (0.05em) node[below left] {\tiny A};
    \draw[solid,thick] (C) -- (B);
    \draw[fill=black] (B) circle (0.05em) node[above] {\tiny B};
    \draw[solid,thick] (A) -- (C);
    \draw[fill=black] (C) circle (0.05em) node[left] {\tiny C};
    \draw[fill=black] (D) circle (0.025em) node[below right] {\tiny D};
    \draw[solid,gray!80] (C) -- (D);

    \draw[<->][thick,gray] ($(O)+(45:4mm)$) arc (45:90:4mm) node[above right] {\tiny $\theta$};

    \draw[solid,gray!80] (B) -- (Bxy);
    \draw[solid,gray!80] (Bxy) -- (O);
    \draw[<->][gray!80] ($(O)+(0:5mm)$) arc (0:17:5mm) node[right] {\tiny {$\phi$}};
    
    \coordinate (AC) at ($(A)!0.5!(C)$);
    \draw[fill=black] (AC) circle (0.025em) node[left] {\tiny e};
    \coordinate (BC) at ($(B)!0.5!(C)$);
    \draw[fill=black] (BC) circle (0.025em) node[above] {\tiny f};
    \coordinate (CD) at ($(C)!0.5!(D)$);
    \draw[fill=black] (CD) circle (0.025em) node[right] {\tiny t};
    \coordinate (AD) at ($(A)!0.5!(D)$);
    \draw[fill=black] (AD) circle (0.025em) node[right] {\tiny g};
	%%% Depending of what you want to display, you can comment/edit
	%%% the following lines

	%% Possibly draw back faces

	%\fill[gray!90] (B2) -- (A2) -- (A3) -- (A4) -- cycle; % face 6
	%\node at (barycentric cs:B2=1,A2=1,A3=1,A4=1) {\tiny n2};
	%% Possibly draw front faces

	% \fill[orange] (B1) -- (A8) -- (A4) -- (B2) -- cycle; % face 1
	% \node at (barycentric cs:B1=1,A8=1,A4=1,B2=1) {\tiny f1};
	%\fill[gray!50,opacity=0.2] (B1) -- (B2) -- (A2) -- (A1) -- cycle; % f2
	%\node at (barycentric cs:B1=1,B2=1,A2=1,A1=1) {\tiny f2};
	%\fill[gray!90,opacity=0.2] (B1) -- (A1) -- (A5) -- (A8) -- cycle; % f5
	%\node at (barycentric cs:B1=1,A1=1,A5=1,A8=1) {\tiny f5};


	%\coordinate (CP) at
	%  (intersection cs: first line={(NE1) -- (NE12)}, 
	%		    second line={(NS2) -- (NE22)});

    %\draw[domain=0:90] (CP) circle (0.16); 
    %\draw[<->] ($(CP)+(-83:2mm)$) arc (-83:23:2mm);

	%\draw[fill=black] (A1) circle (0.05em)
	%    node[above right] {\tiny 1};
	%\draw[fill=black] (t1) circle (0.05em)
	%    node[right] {\tiny x1};
	%\draw[fill=black] (t2) circle (0.05em)
	%    node[right] {\tiny x2};
	% \draw[fill=black] (P1) circle (0.1em) node[below] {\tiny p1};
	% \draw[fill=black] (P2) circle (0.1em) node[below] {\tiny p2};
\end{tikzpicture}
\end{document} 
